
% 10pt is the smallest font for article
\documentclass{article}

% \usepackage{graphicx}
\usepackage{epsf}
\usepackage{a4}
\usepackage{palatino}
\usepackage{euler}
\newcommand {\atilde} {$_{\char '176}$} % tilde(~) character

\title{Coot Tutorial II: More Advanced Usage}
%(Part of Protein Crystallography)}

%\author{Your Workshop Here}
%\author{M.Res. Functional Genomics}
%\author{CCP4 Workshop Bangalore}
%\author{CCP4 Workshop}
%\author{CSHL 2006 Workshop}
%\author{Osaka 2006 Workshop}
%\author{EMBO 2007 Workshop}
\author{Next 2008 Workshop}

\begin{document}
\maketitle
%\tableofcontents
%\listoffigures

The idea here is to use more advanced\footnote{``less commonly-used''
  might be a better description} tools of Coot.  There will be less
description of low-level widget manipulation in this tutorial - we
presume that you already have experience with that.

When automatic building fails, typically because the resolution limit
of your data is too low, then building the molecule ``by hand'' may be
the only way to proceed.  Recognising the shape of main-chain and
side-chain densities is valuable and this tutorial aims to introduce
these to you.  Note that this tutorial map is an easy map to build
into, the sidechains are (mostly) clear.  If you want a more realistic
``bad'' map, you can apply a resolution limit to the data read in from
the MTZ file\footnote{the resolution limit widget will appear when you
  activate the ``Expert Mode`` button.}.

Using just a map and a sequence, we will attempt to generate a model.
This model can then be validatated and refined with Refmac for several
rounds.  With some experience you should be able to get an R-factor of
less than 20\% in less than 30 minutes.

\section{Skeletonization and Baton Building}


You can calculate the map skeleton in Coot directly:

\textsf{Calculate $\rightarrow$ Map Skeleton\ldots $\rightarrow$ On.}

This can be used to ``baton build'' a map.  You can turn off the
coordinates and try it if you like (the Baton Building window can be
found by clicking \textsf{``Ca Baton Mode\ldots''} in the Other
Modelling Tools dialog.
  
I suggest you use Go To Atom and start residue 2 A. This allows you to
build the complete A chain in the correct direction and you can
directly compare it to the real structure afterwards\footnote{if don't
  follow this instruction, you could well build a symmetry releted
  molecule, which is perfectly valid, of course, just that the
  comparison versus the correct structure will be more difficult.}.
Once you are at residue 2A, use the Display Manager to turn off the
{\small\texttt{``tutorial-modern.pdb''}} and don't look at it again
until you have finished building, validating and refining.
  
Remember, when you start, you are placing a CA at the baton
\emph{tip} and at the start you are placing atom CA 1.  This might
seem that you are ``double-backing'' on yourself - which can be
confusing the first time.

So build from the N-terminus to the C (it takes about 15 minutes or
so).  There are 96 residues to build.

\section{Key Bindings}

If you look at "Paul's Key Bindings"\footnote{Use Bernhard's
  Keybindings if you are using pythonized or WinCoot} in
the Coot Wiki\footnote{you can find a link to this from the Coot web
  page}, you will see a page of customizations.  One of those
customizations can help you in Baton-Building mode - and that is the
``quoteleft'' key binding.

So, cut the bindings out of the web page, paste them into a file and
then use \textsf{Calculate $\rightarrow$ Run Script\ldots} to evaluate
that file\footnote{``read it in'', you might say}.  To check that your
keybindings are activated, Use \textsf{Extensions $\rightarrow$ Key
  Bindings\ldots}.

Now, we can use quoteleft (or ``backquote'', "`" is how it might
appear on the keyboard) to accept the batton position - this is much
more convenient than using the ``Accept'' button\footnote{You can do
  that as well, of course, but \emph{clicky-clicky pressy button} is for
Coot noobs, and that's not us, right?}.


\section{At the end of the Chain} 

At some stage\footnote{hopefully residue 96} you will come to a point
where no progress can be made, the only direction takes us into
density we've already built into.  OK, so stop: \textsl{Dismiss}.

Now we need to turn these CA positions into mainchain.
\textsf{Calculate $\rightarrow$ Other Modelling Tools $\rightarrow$ CA
  Zone to Mainchain}.  Use the \textsf{Go To Atom} dialog to centre on the
first residue of ``Baton Atoms'', click it, then centre on the last
residue of ``Baton Atoms'' and click on that.

\textsl{  [Coot thinks for a several seconds while building a mainchain]}

OK, great, we have a mainchain.  Let's tidy it up.

\textsl{Extensions $\rightarrow$ Stepped Refine}.  

Refine the ``mainchain'' molecule, watch it as it goes.  Is it making
mistakes?

That refinement may have gone to quickly to make a note of problem areas, so use 
\textsl{Validation $\rightarrow$ Density Fit Analysis} on the
``mainchain'' molecule and find areas that are marked with large spikes.  

``There are none'' you say?  Good\footnote{If that's not what you say,
  you can use the refinement or other tools that we learnt about in
  the first tutorial to improve the fit to density.}. Let's move on.

\section{Assign Sequence}

Let's tell Coot that we have a sequence assocciated with this set
of CA points.  So, \textsl{Extensions $\rightarrow$ Dock Sequence
  $\rightarrow$ Assign Sequence}

Turn on auto-fit of residues

So when the file is assigned ``Assign Closest fragment''.

\textsl{ [Coot thinks for a several seconds while assigning
  sidechains, then goes about mutating and fitting the residues]}

What's that you say?  Coot didn't do that?  Well, that's because you
mainchain model is too bad for Coot to recognise the sidechain
positions.  You need to review you mainchain model and make sure sure
that the CBs are in density and pointing in the right direction.  When
you have improved you model sufficiently well, Coot will apply the
sequence to it using the above method.

Change the Chain ID from `` `` to ``A''.

\section{Cell and Symmetry}

Check for unassignded density - by eyeballing.

\section{Build another molecule}

or part of it.

Apply seqence.

Merge molecules.

Tell Coot there is NCS (or get it to regenerate it).  Show ghosts.  

If ghosts appear, use NCS residue range from A to B.

stepped refine on the B chain.

Now unmodelled blobs - like we did before.

Find the ligand 3GP, merge it in, 

Refine.  Validate. Rebuild.


\end{document}
